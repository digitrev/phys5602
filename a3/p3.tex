\begin{enumerate}
	\item The minimum energy particle $B$ can have is when $B$ is at rest, and hence $E_{\text{min}} = m_B$. The maximum energy particle $B$ can have is when the other particles are at rest, and their energies are $m_C, m_D$. Then the available energy is
	\begin{align*}
		E_{\text{max}} &= m_A - m_B - m_C - m_D + m_B\\
			&= m_A - m_C - m_D
	\end{align*}
	
	And hence $m_B \leq E_B \leq m_A - m_C - m_D$. Note that all of this assumes that $m_A \geq m_B + m_C + m_D$.

	\item Taking neutrino mass to be zero, we therefore get $m_e \leq E_e \leq m_\mu$, or $\SI{511}{keV} \leq E_e \leq \SI{106}{MeV}$.
\end{enumerate}