For $\nu_\mu + e^- \to \nu_\mu + e^-$, the Feynman diagram is

\begin{center}
\begin{fmffile}{numu}
	\begin{fmfgraph*}(40,25)
		\fmfleft{i1,i2}
		\fmfright{o1,o2}

		\fmf{fermion,label=$e^-$}{i1,v1,i2}
		\fmf{fermion,label=$\nu_\mu$}{o1,v2,o2}
		\fmf{boson,label=$Z^0$}{v1,v2}

%		\fmflabel{$e^-$}{i1}
%		\fmflabel{$e^-$}{i2}
%		\fmflabel{$\nu_\mu$}{o1}
%		\fmflabel{$\nu_\mu$}{o2}
	\end{fmfgraph*}
\end{fmffile}
\end{center}

and for $\nu_e + e^- \to \nu_e + e^-$, the Feynman diagram is

\begin{center}
	\begin{fmffile}{nue}
		\begin{fmfgraph*}(40,25)
			\fmfleft{i1,i2}
			\fmfright{o1,o2}
			
			\fmf{fermion,label=$e^-$}{i1,v1,i2}
			\fmf{fermion,label=$\nu_e$}{o1,v2,o2}
			\fmf{boson,label=$Z^0$}{v1,v2}
			
			%		\fmflabel{$e^-$}{i1}
			%		\fmflabel{$e^-$}{i2}
			%		\fmflabel{$\nu_\mu$}{o1}
			%		\fmflabel{$\nu_\mu$}{o2}
		\end{fmfgraph*}
	\end{fmffile}
\end{center}

The first interaction is a more unequivocal demonstration of weak neutral currents. This is because there is actually a second possible diagram for the second interaction: 

\begin{center}
	\begin{fmffile}{nuecharged}
		\begin{fmfgraph*}(40,25)
			\fmfleft{i1,i2}
			\fmfright{o1,o2}
			
			\fmf{fermion,label=$e^-$}{i1,v1}
			\fmf{fermion,label=$e^-$}{v2,o2}
			\fmf{fermion,label=$\nu_e$}{o1,v2}
			\fmf{fermion,label=$\nu_e$}{v1,i2}
			\fmf{boson,label=$W^-$}{v1,v2}
			
			%		\fmflabel{$e^-$}{i1}
			%		\fmflabel{$e^-$}{i2}
			%		\fmflabel{$\nu_\mu$}{o1}
			%		\fmflabel{$\nu_\mu$}{o2}
		\end{fmfgraph*}
	\end{fmffile}
\end{center}

This also conserves lepton number and charged, but demonstrates weak \emph{charged} current, as opposed to weak \emph{neutral} current.
