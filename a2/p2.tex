\begin{enumerate}
	\item $\nu_\mu + p \to \mu^+ + n$
	
	Lepton number is not conserved, since $L(\nu_\mu) = +1$ and $L(\mu^+) = -1$. Therefore, this interaction is forbidden.
	\item $\Sigma^0 \to \pi^+ + e^- + \bar{\nu}_e$
	
	Since $\Sigma^0$ is a baryon, and $\pi^+$ is a meson, baryon number is not conserved. Therefore, this interaction is forbidden.
	\item $\pi^0\to \gamma + \gamma$
	
	Baryon number is conserved. Lepton number is conserved. Charge is conserved. Mass will be conserved since the photons will take all the energy. Momentum will be conserved, for the same reason. Therefore, this is a possible reaction.
	\item $p\to n + e^+ + \nu_e$
	
	The mass of the proton is $\SI{938.27}{MeV}$ and the mass of the neutron is $\SI{939.57}{MeV}$. Thus, this will only occur when the proton has sufficient energy to create the final particles. All other conservation laws are obeyed.
	\item $\tau^+ \to e^+ + e^- + e^+$
	
	Lepton number is not conserved, and hence this is a forbidden interaction.
	\item $p + \bar{p} \to \pi^+ + \pi^- + \pi^0$
	
	Charge is conserved. Lepton number is conserved. Baryon number is conserved (since the baryon number of an anti-baryon is negative one). The mass of the proton/anti-proton is \SI{939.27}{MeV}. The mass of each charged pion is \SI{139.57}{MeV}, and the uncharged pion's mass is \SI{134.98}{MeV}. Thus, mass can be conserved, as can momentum. Therefore, this is a possible reaction.
	\item $\pi^+ \to e^+ + \nu_e$
	
	Charge, lepton number, and baryon number are all conserved. Mass will be conserved, since $m_{\pi^+} > m_{e^+}$, and so will momentum. Thus, this is a possible reaction. 
	\item $K^+ \to \pi^+ + \pi^- + \gamma$
	
	Charge is not conserved, and hence this reaction is impossible. 
	\item $\nu_e + \bar{\nu}_e \to g$
	
	Since leptons do not carry color, this is an impossible reaction.
	\item $Z^0 \to t + \bar{t}$
	
	This is forbidden by conservation of mass, unless the $Z^0$ has enough energy to create the top and anti-top particles. Namely, the $Z^0$ has a mass of \SI{91.2}{GeV}, whereas the $t$ quark (and hence the $\bar{t}$) has a mass of \SI{172.9}{GeV}. 
\end{enumerate}
